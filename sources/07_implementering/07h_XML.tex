
\documentclass[../../main.tex]{subfiles}

\begin{document}

\subsection{XML}
En vigtig ting i spil og andre applikationer nu til dags er konfiguration. Man skal kunne ændre nogle ting i sit program uden at ændre i sin kildekode. Dette gælder i vores tilfælde eksempelvist oversættelse, spillebrættet og chancekort. Vi har valgt at tage brug af XML (Extensible Markup Language) til at læse fra. 

\subsubsection{Hvad er XML?}
XML er et markdown-sprog, der bruger tags meget lig HTML. Hvert åbningstag kan indeholde attributter, og mellemrummet mellem åbnings- og lukningstags kan indeholde flere tags eller almindelig tekst. Ethvert XML-dokument har brug for et rodtag, der indeholder alle andre tags.


\begin{lstlisting}[escapeinside={(*}{*)},language=XML,numbers=left]
<?xml version="1.0" encoding="utf-8" ?>
<root-tag> 
  <tag attribute="some value"> content </tag>
</root-tag>
\end{lstlisting}
\end{document}