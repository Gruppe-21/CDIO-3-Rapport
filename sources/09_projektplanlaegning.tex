\documentclass[../main.tex]{subfiles}

\begin{document}

\section{Projektplanlægning}

%Indledning af projekt-afsnittet
\subsection{Udviklingsproces for forløb}
\begin{flushleft}
\todo [Skal have mindre omskrivninger - se evt. CDIO2] \newline
   For dette projekt har vi primært haft den samme strategi for udarbejdelse af projektet, som i CDIO-2. Her har vi taget udgangspunkt i Unified Proces (UP), som vores udvilkingsproces. Som nævnt i indledningen dækker UP over at opdele projektet i mindre delelementer, kaldet iterationer. Disse iterationer bliver samlet løbende, hvilket kaldes at projektet vil vokse iterativt - altså at projeketet vokser stødt, efter hver fuldførte delelement. UP kan stilles op i en model, som viser de forskellige faser, descipliner og iterationer for UP-projekt:
\end{flushleft}

%Beskrivelse af vores projektplanlægning
\subsection{Planlagte forløb}
\begin{flushleft}
   tekst
\end{flushleft}

%Beskrivelser af faktiske forløb.
\subsection{Faktiske forløb}
\begin{flushleft}
   tekst
\end{flushleft}

\todo Måske skrive at vi har haft kontakt med kunden, som en del af vores projektplanglægning

\end{document}